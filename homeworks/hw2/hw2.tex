
\documentclass[10pt]{article}

\usepackage{graphicx}
\usepackage{amsmath,amsfonts,amssymb}

\usepackage{hyperref}  % for urls and hyperlinks


\setlength{\textwidth}{6.2in}
\setlength{\oddsidemargin}{0.3in}
\setlength{\evensidemargin}{0in}
\setlength{\textheight}{8.9in}
\setlength{\voffset}{-1in}
\setlength{\headsep}{26pt}
\setlength{\parindent}{0pt}
\setlength{\parskip}{5pt}


\input{../latex/macros.tex}


\begin{document}

% header:
\hfill \vbox{
\hbox{AMath 584 / Math 584}
\hbox{Homework \#2}
\hbox{Due 11:00pm PDT}
\hbox{Tuesday, October 25, 2016}
}


\vskip 0.5cm

{\bf Name:}   Your Name Here

{\bf Netid:}  Your NetID Here

\vskip 0.5cm

%--------------------------------------------------------------------------
\vskip 1cm
\hrule
{\bf Problem 1.}
Find (by hand) both the reduced and full SVD of the matrix
\[
A = \bcm 0&3\\ 2&0\\ 0&4\ecm.
\]
Remember that things should be ordered so that $\sigma_1 \geq \sigma_2$.

Use the SVD to determine the following:
\begin{itemize} 
\item The best rank-1 approximation to $A$.
\item The 2-norm of $A$.
\item A basis for the linear subspace $\{x \in \reals^2: 
      \|Ax\|_2 = \|A\|_2 \|x\|_2\}$ and the dimension of this space.
\item A basis for the orthogonal complement of the range of $A$.
\item A basis for the null space of $A^T$.
\end{itemize} 

% uncomment the next two lines if you want to insert solution...
%\vskip 1cm
%{\bf Solution:}

% insert your solution here!


%--------------------------------------------------------------------------
\vskip 1cm
\hrule
{\bf Problem 2.}
Determine by hand the SVD of the matrix
\[
A = \bcm 17&1\\ 6&18\ecm.
\]
Hint: one right singular vector is 
\[
v_1 = \bcm 1/\sqrt{2}\\ 1/\sqrt{2} \ecm.
\]
 

% uncomment the next two lines if you want to insert solution...
%\vskip 1cm
%{\bf Solution:}

% insert your solution here!


%--------------------------------------------------------------------------
\vskip 1cm
\hrule
{\bf Problem 3.}
Exercise 6.1 in Trefethen and Bau.  In addition, 
illustrate with a sketch showing the effect of $A = I-2P$ on a typical vector
$v$ for the case
\[
P = \frac 1 5 \bcm 1&2\\2&4\ecm.
\]
Also explain why a matrix $A$ of this form is sometimes called a
``reflector''.  {\bf Hint:} See also Figure 10.2 in the book and the
discussion for the case of a ``Householder reflector''.


% uncomment the next two lines if you want to insert solution...
%\vskip 1cm
%{\bf Solution:}

% insert your solution here!

%--------------------------------------------------------------------------
\vskip 1cm
\hrule
{\bf Problem 4.}
Exercise 6.3 in Trefethen and Bau.  Use the SVD to do this.


% uncomment the next two lines if you want to insert solution...
%\vskip 1cm
%{\bf Solution:}

% insert your solution here!


%--------------------------------------------------------------------------
\vskip 1cm
\hrule
{\bf Problem 5.}
Exercise 6.4 in Trefethen and Bau.


% uncomment the next two lines if you want to insert solution...
%\vskip 1cm
%{\bf Solution:}

% insert your solution here!


%--------------------------------------------------------------------------
\vskip 1cm
\hrule
{\bf Problem 6.}
Exercise 7.1 in Trefethen and Bau.


% uncomment the next two lines if you want to insert solution...
%\vskip 1cm
%{\bf Solution:}

% insert your solution here!

%--------------------------------------------------------------------------
\vskip 1cm
\hrule
{\bf Problem 7.}
Exercise 7.2 in Trefethen and Bau.


% uncomment the next two lines if you want to insert solution...
%\vskip 1cm
%{\bf Solution:}

% insert your solution here!

%--------------------------------------------------------------------------
\vskip 1cm
\hrule
{\bf Problem 8.}
Exercise 8.2 in Trefethen and Bau.  

You can write a Python function instead of Matlab if you prefer.  

Test your routine on the matrices from Exercise 6.4 of Trefethen and Bau  
and submit your output.

Test it on other matrices of different sizes to convince yourself it is
working properly. You do not need to turn in more output, but submit the
code in a file {\tt mgs.m} or {\tt mgs.py} in a way that we can test it on other
matrices of our choosing.

Please write readable code!


% uncomment the next two lines if you want to insert solution...
%\vskip 1cm
%{\bf Solution:}

% insert your solution here!

\end{document}

